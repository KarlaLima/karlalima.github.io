\documentclass[a4paper,5pt]{amsbook}
%%%%%%%%%%%%%%%%%%%%%%%%%%%%%%%%%%%%%%%%%%%%%%%%%%%%%%%%%%%%%%%%%%%%%

\usepackage{booktabs}
\usepackage{graphicx}
% \usepackage[]{float}
\usepackage{amssymb}
% \usepackage{amsfonts}
% \usepackage[]{amsmath}
% \usepackage[]{epsfig}
% \usepackage[brazil]{babel}
\usepackage[utf8]{inputenc}
% \usepackage{verbatim}
%\usepackage[]{pstricks}
%\usepackage[notcite,notref]{showkeys}
\usepackage{subcaption}

%%%%%%%%%%%%%%%%%%%%%%%%%%%%%%%%%%%%%%%%%%%%%%%%%%%%%%%%%%%%%%

\newcommand{\sen}{\text{sen}}
\newcommand{\ds}{\displaystyle}

%%%%%%%%%%%%%%%%%%%%%%%%%%%%%%%%%%%%%%%%%%%%%%%%%%%%%%%%%%%%%%%%%%%%%%%%

\setlength{\textwidth}{16cm} %\setlength{\topmargin}{-0.1cm}
\setlength{\leftmargin}{1.2cm} \setlength{\rightmargin}{1.2cm}
\setlength{\oddsidemargin}{0cm}\setlength{\evensidemargin}{0cm}

%%%%%%%%%%%%%%%%%%%%%%%%%%%%%%%%%%%%%%%%%%%%%%%%%%%%%%%%%%%%%%%%%%%%%%%%

% \renewcommand{\baselinestretch}{1.6}
% \renewcommand{\thefootnote}{\fnsymbol{footnote}}
% \renewcommand{\theequation}{\thesection.\arabic{equation}}
% \setlength{\voffset}{-50pt}
% \numberwithin{equation}{chapter}

%%%%%%%%%%%%%%%%%%%%%%%%%%%%%%%%%%%%%%%%%%%%%%%%%%%%%%%%%%%%%%%%%%%%%%%

\begin{document}
\thispagestyle{empty}
\hspace{-0.6cm}
\begin{minipage}[p]{0.14\linewidth}
	\includegraphics[scale=0.24]{ufgd.png}
\end{minipage}
\begin{minipage}[p]{0.7\linewidth}
\begin{tabular}{c}
\toprule{}
{{\bf UNIVERSIDADE FEDERAL DA GRANDE DOURADOS}}\\
{{\bf Profª.\ Karla Lima}}\\
\midrule{}
{{\bf Cálculo II}}\hspace{4cm}22 de Setembro de 2017 \\



\bottomrule{}
\end{tabular}
\vspace{-0.45cm}
\end{minipage}
\vspace{1 cm}
%------------------------
%%%%%%%%%%%%%%%%%%%%%%%%%%%%%%%%   formulario  inicio  %%%%%%%%%%%%%%%%%%%%%%%%%%%%%%%%
\begin{enumerate}
\item Calcule a integral fazendo a substituição dada.
\begin{itemize}
 \item[a)]$\displaystyle\int_{1}^{2}\dfrac{dx}{(3-5x)^2}$, $u=3-5x$.\\
 \item[b)]$\displaystyle\int_{0}^{\pi}\cos(3x)dx$, $u=3x$.\\
 \item[c)]$\displaystyle\int_{0}^{1}x(4+x^2)^{10}dx$, $u=4+x^2$.\\
 \item[d)]$\displaystyle\int_{0}^{\pi/2}\cos^3\theta \sen\theta d\theta$, $u=\cos\theta$.\\
 \item[e)]$\displaystyle\int_{0}^{1}(x^2-1)^4x^5 dx$, $u=x^2-1$.\\
\end{itemize}
\item Avalie a integral definida.
\begin{itemize}
 \item[a)]$\displaystyle\int_{0}^{1}\cos(\pi t/2)dt$.\\
 \item[b)]$\displaystyle\int_{1}^{2}\dfrac{e^{1/x}}{x^2}dx$.\\
 \item[c)]$\displaystyle\int_{e}^{e^4}\dfrac{dx}{x\sqrt{\ln x}}dx$.\\
 \item[d)]$\displaystyle\int_{0}^{1}\dfrac{e^z+1}{e^z+z}dz$.\\
 \item[e)]$\displaystyle\int_{0}^{1}\dfrac{dx}{1+\sqrt{x}}$.\\
\end{itemize}


\end{enumerate}
\begin{center}
 \textbf{Gabarito}
\end{center}
\begin{enumerate}
\item 
\begin{itemize}
 \item[a)]$\dfrac{1}{14}$\\
 \item[b)]$0$
 \item[c)]$\dfrac{5^{11}-4^{11}}{22}$
 \item[d)]$\dfrac{1}{4}$\\
 \item[e)]$\dfrac{1}{210}$\\
\end{itemize}
\item 
\begin{itemize}
 \item[a)]$\dfrac{2}{\pi}$
 \item[b)]$e-\sqrt{e}$
 \item[c)]$2$
 \item[d)]$\ln(e+1)$
 \item[e)]$2-2\ln2$
\end{itemize}


\end{enumerate}
\end{document}
